\documentclass[12pt]{article}
\usepackage[top=1in, bottom=1in, left=1in, right=1in]{geometry}

\usepackage{setspace}
\onehalfspacing

\usepackage{amssymb}
%% The amsthm package provides extended theorem environments
\usepackage{amsthm}
\usepackage{epsfig}
\usepackage{times}
\renewcommand{\ttdefault}{cmtt}
\usepackage{amsmath}
\usepackage{graphicx} % for graphics files
\usepackage{tabu}

% Draw figures yourself
\usepackage{tikz} 

% writing elements
%\usepackage{mhchem}

\usepackage{paralist}

% The float package HAS to load before hyperref
\usepackage{float} % for psuedocode formatting
\usepackage{xspace}

% from Denovo Methods Manual
\usepackage{mathrsfs}
\usepackage[mathcal]{euscript}
\usepackage{color}
\usepackage{array}

\usepackage[pdftex]{hyperref}
\usepackage[parfill]{parskip}

% math syntax
\newcommand{\nth}{n\ensuremath{^{\text{th}}} }
\newcommand{\ve}[1]{\ensuremath{\mathbf{#1}}}
\newcommand{\Macro}{\ensuremath{\Sigma}}
\newcommand{\rvec}{\ensuremath{\vec{r}}}
\newcommand{\vecr}{\ensuremath{\vec{r}}}
\newcommand{\omvec}{\ensuremath{\hat{\Omega}}}
\newcommand{\vOmega}{\ensuremath{\hat{\Omega}}}
\newcommand{\even}{\ensuremath{\phi^g}}
\newcommand{\odd}{\ensuremath{\vartheta^g}}
\newcommand{\evenp}{\ensuremath{\phi^{g'}}}
\newcommand{\oddp}{\ensuremath{\vartheta^{g'}}}
\newcommand{\Sn}{\ensuremath{S_N} }
\newcommand{\Ye}[2]{\ensuremath{Y^e_{#1}(\vOmega_#2)}}
\newcommand{\Yo}[2]{\ensuremath{Y^o_{#1}(\vOmega_#2)}}
\newcommand{\sigg}[1]{\ensuremath{\Macro^{gg'}_{s\,#1}}}
\newcommand{\psig}{\ensuremath{\psi^g}}
\newcommand{\Di}{\ensuremath{\Delta_i}}
\newcommand{\Dj}{\ensuremath{\Delta_j}}
\newcommand{\Dk}{\ensuremath{\Delta_k}}
%---------------------------------------------------------------------------
%---------------------------------------------------------------------------
\begin{document}
\begin{center}
{\bf NE 255, Fa16 \\
Eigenvalue Formulation and Solutions\\
October 27, 2016}
\end{center}

\setlength{\unitlength}{1in}
\begin{picture}(6,.1) 
\put(0,0) {\line(1,0){6.25}}         
\end{picture}

We're going to start talking about how to find the criticality state of a reactor, or the dominant eigenvalue-eigenvector pair. First, we're going to talk generally about eigenvalue solution methods, and then apply them to the transport equation. 

\subsection*{Background}
A right eigenvector is defined as a column vector $x_R$ satisfying
\[\ve{A}x_R=\lambda_R x_R\:.\] 
In many common applications, only right eigenvectors (and not left eigenvectors) need be considered. Hence the unqualified term ``eigenvector" can be understood to refer to a right eigenvector. %http://mathworld.wolfram.com/RightEigenvector.html

A left eigenvector is defined as a row vector $x_L$ satisfying
\[ x_L\ve{A}=\lambda_L x_L\:.\] %http://mathworld.wolfram.com/LeftEigenvector.html

The \textbf{generalized eigenvalue problem} takes the form $\ve{B}x = \mu \ve{C}x$ and can be transformed into an \textbf{ordinary eigenvalue problem}, $\ve{A}x = \lambda x$. Both forms have the same right eigenvectors. If $\ve{C}$ is non-singular then $\ve{A} = \ve{C}^{-1}\ve{B}$ and the problem $v = \ve{A}x$ can be solved in two steps: %\cite{Stewart2001}
%
\begin{enumerate}
  \item $w = \ve{B}x$
  \item Solve the system $\ve{C}v = w$.
\end{enumerate}
%
Because the generalized form can be converted to the ordinary form, we will focus on the more common ordinary form without loss of applicability.

Recall this basic notation: let $\sigma(\ve{A}) \equiv \{\lambda \in \mathbb{C} : rank(\ve{A} - \lambda \ve{I}) \le n\}$ be the spectrum of $\ve{A}$, where the elements in the set are the eigenvalues and $\mathbb{C}$ is the set of complex numbers. The eigenvalues can be characterized as the $n$ roots of the polynomial $p_{\ve{A}}(\lambda) \equiv det(\lambda \ve{I} - \ve{A})$. Each distinct eigenvalue in $\sigma(\ve{A})$ has a corresponding nonzero vector $x$ such that $\ve{A}x_{i} = \lambda_{i} x_{i}$ for $i = 1,...,n$. % \cite{Sorensen1996}. 
It will be assumed that the eigenvalues are ordered as $|\lambda_{1}| > |\lambda_{2}| \ge \dots \ge |\lambda_{n}| \ge 0$. 

Eigenvalue problems in the nuclear transport community are typically solved with iterative rather than direct methods. A variety of iterative solvers have been used to solve eigenvalue problems. These are ones that have been most widely used. 


\end{document}
